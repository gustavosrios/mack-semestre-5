%% abtex2-modelo-artigo.tex, v-1.9.7 laurocesar
%% Copyright 2012-2018 by abnTeX2 group at http://www.abntex.net.br/ 
%%
%% This work may be distributed and/or modified under the
%% conditions of the LaTeX Project Public License, either version 1.3
%% of this license or (at your option) any later version.
%% The latest version of this license is in
%%   http://www.latex-project.org/lppl.txt
%% and version 1.3 or later is part of all distributions of LaTeX
%% version 2005/12/01 or later.
%%
%% This work has the LPPL maintenance status `maintained'.
%% 
%% The Current Maintainer of this work is the abnTeX2 team, led
%% by Lauro César Araujo. Further information are available on 
%% http://www.abntex.net.br/
%%
%% This work consists of the files abntex2-modelo-artigo.tex and
%% abntex2-modelo-references.bib
%%

% ------------------------------------------------------------------------
% ------------------------------------------------------------------------
% abnTeX2: Modelo de Artigo Acadêmico em conformidade com
% ABNT NBR 6022:2018: Informação e documentação - Artigo em publicação 
% periódica científica - Apresentação
% ------------------------------------------------------------------------
% ------------------------------------------------------------------------

\documentclass[
	% -- opções da classe memoir --
	article,			% indica que é um artigo acadêmico
	11pt,				% tamanho da fonte
	oneside,			% para impressão apenas no recto. Oposto a twoside
	a4paper,			% tamanho do papel. 
	% -- opções da classe abntex2 --
	%chapter=TITLE,		% títulos de capítulos convertidos em letras maiúsculas
	%section=TITLE,		% títulos de seções convertidos em letras maiúsculas
	%subsection=TITLE,	% títulos de subseções convertidos em letras maiúsculas
	%subsubsection=TITLE % títulos de subsubseções convertidos em letras maiúsculas
	% -- opções do pacote babel --
	english,			% idioma adicional para hifenização
	brazil,				% o último idioma é o principal do documento
	sumario=tradicional
	]{abntex2}
\usepackage[utf8]{inputenc}

% ---
% PACOTES
% ---

% ---
% Pacotes fundamentais 
% ---
\usepackage{lmodern}			% Usa a fonte Latin Modern
\usepackage[T1]{fontenc}		% Selecao de codigos de fonte.
\usepackage[utf8]{inputenc}		% Codificacao do documento (conversão automática dos acentos)
\usepackage{indentfirst}		% Indenta o primeiro parágrafo de cada seção.
\usepackage{nomencl} 			% Lista de simbolos
\usepackage{color}				% Controle das cores
\usepackage{graphicx}			% Inclusão de gráficos
\usepackage{microtype} 			% para melhorias de justificação
\usepackage[portuguese]{babel}
% ---

        
% ---
% Pacotes adicionais, usados apenas no âmbito do Modelo Canônico do abnteX2
% ---
\usepackage{lipsum}				% para geração de dummy text
% ---
		
% ---
% Pacotes de citações
% ---
\usepackage[portuguese]{babel}
\usepackage[brazilian,hyperpageref]{backref}	 % Paginas com as citações na bibl
\usepackage[alf]{abntex2cite}	% Citações padrão ABNT
% ---

% ---
% Configurações do pacote backref
% Usado sem a opção hyperpageref de backref
%\renewcommand{\backrefpagesname}{Citado na(s) página(s):~}
% Texto padrão antes do número das páginas
%\renewcommand{\backref}{}
% Define os textos da citação
%\renewcommand*{\backrefalt}[4]{
%	\ifcase #1 %
%		Nenhuma citação no texto.%
%	\or
%		Citado na página #2.%
%	\else
%		Citado #1 vezes nas páginas #2.%
%	\fi}%
% ---

% --- Informações de dados para CAPA e FOLHA DE ROSTO ---


\titulo{Alimentando Informações: Um Estudo De Dados Nutricionais}
\def \theforeigntitle{}

\autor{
\textsuperscript{} Gustavo Silva Rios\\ Silas de Souza Ferreira \\ Israel Soares do N. Viana\\
\\
\\
\textsuperscript{}Faculdade de Computação e Informática (FCI) \\ Universidade Presbiteriana Mackenzie São Paulo, SP – Brasil \\ 
\\
\\
}

\data{2025}
% ---

% ---
% Configurações de aparência do PDF final

% alterando o aspecto da cor azul
\definecolor{blue}{RGB}{41,5,195}

% informações do PDF
\makeatletter
\let\@fnsymbol\@arabic
\hypersetup{
     	%pagebackref=true,
		%pdftitle={\@title}, 
		%pdfauthor={\@author},
    	%pdfsubject={Modelo de artigo científico com abnTeX2},
	    %pdfcreator={LaTeX with abnTeX2},
		%pdfkeywords={abnt}{latex}{abntex}{abntex2}{atigo científico}, 
		colorlinks=true,       		% false: boxed links; true: colored links
    	linkcolor=blue,          	% color of internal links
    	citecolor=blue,        		% color of links to bibliography
    	%filecolor=magenta,         % color of file links
		urlcolor=blue,
		%bookmarksdepth=4
}
\makeatother
% --- 

% ---
% compila o indice
% ---
\makeindex
% ---

% ---
% Altera as margens padrões
% ---
\setlrmarginsandblock{3cm}{3cm}{*}
\setulmarginsandblock{3cm}{3cm}{*}
\checkandfixthelayout
% ---

% --- 
% Espaçamentos entre linhas e parágrafos 
% --- 

% O tamanho do parágrafo é dado por:
\setlength{\parindent}{1.3cm}

% Controle do espaçamento entre um parágrafo e outro:
\setlength{\parskip}{0.2cm}  % tente também \onelineskip

% Espaçamento simples
\SingleSpacing


\usepackage[brazil]{babel}

% ----
% Início do documento
% ----
\begin{document}

% Seleciona o idioma do documento (conforme pacotes do babel)
%\selectlanguage{english}
\selectlanguage{brazil}

% Retira espaço extra obsoleto entre as frases.
\frenchspacing 

% ----------------------------------------------------------
% ELEMENTOS PRÉ-TEXTUAIS
% ----------------------------------------------------------

%---
%
% Se desejar escrever o artigo em duas colunas, descomente a linha abaixo
% e a linha com o texto ``FIM DE ARTIGO EM DUAS COLUNAS''.
% \twocolumn[    		% INICIO DE ARTIGO EM DUAS COLUNAS
%
%---

% página de titulo principal (obrigatório)
\maketitle
% titulo em outro idioma (opcional)



% resumo em português
\begin{resumoumacoluna}
\selectlanguage{brazil}
\textit{}
Este projeto tem como objetivo analisar os dados do Sistema de Vigilância Alimentar e Nutricional (\textit{Sisvan}) para compreender padrões do estado nutricional da população atendida na Atenção Primária à Saúde (APS). A relevância do estudo se dá pela necessidade de fornecer \textit{insights} para profissionais de saúde e gestores públicos, auxiliando na formulação de políticas eficazes contra desnutrição e obesidade.

A base de dados utilizada contém informações detalhadas sobre medidas antropométricas, dados demográficos e classificações nutricionais de indivíduos desde o ano de 2008. Para garantir uma análise abrangente e longitudinal, os registros do ano de 2018 a 2023 serão integrados, criando uma base consolidada e permitindo o estudo de tendências ao longo do tempo.

Os principais objetivos incluem a exploração da estrutura dos dados, análise de distribuições por idade e região, desenvolvimento de modelos preditivos, detecção de outliers e criação de visualizações interativas. A partir disso, espera-se identificar grupos vulneráveis, entender e propor melhorias para o monitoramento nutricional da população brasileira.
 
 \noindent
\end{resumoumacoluna}
% resumo em inglês

% ]  				% FIM DE ARTIGO EM DUAS COLUNAS
% ---

% ----------------------------------------------------------
% ELEMENTOS TEXTUAIS
% ----------------------------------------------------------
\textual

% ----------------------------------------------------------
% Introdução
% ----------------------------------------------------------

\section{Introdução}
\selectlanguage{brazil}
% Codificacao do documento (conversão automática dos acentos)
\textit{}
A análise de dados em saúde pública desempenha um papel fundamental na formulação de políticas e no aprimoramento de estratégias para a promoção do bem-estar da população. No Brasil, o Sistema de Vigilância Alimentar e Nutricional (\textit{Sisvan}), gerenciado pelo Ministério da Saúde, constitui uma ferramenta estratégica para o acompanhamento do estado nutricional e dos hábitos alimentares da população atendida na Atenção Primária à Saúde (APS). A base de dados do \textit{Sisvan} consolida informações individualizadas e anonimizadas, permitindo a avaliação de indicadores nutricionais ao longo dos anos e fornecendo subsídios para o desenvolvimento de estratégias voltadas à promoção da saúde e à prevenção de doenças associadas à alimentação.

Este trabalho tem como objetivo explorar e manipular os dados provenientes do \textit{Sisvan}, identificando padrões e tendencias relacionadas ao estado nutricional da população brasileira. A análise será conduzida a partir de registros de antropometria e consumo alimentar coletados desde 2008, considerando a integração desses dados com outras fontes, como o \textit{e-SUS APS}  e o Programa Auxílio Brasil (anteriormente conhecido como Bolsa Família).

Ao aplicar métodos avançados de tratamento e análise de dados, é possível extrair \textit{insights} relevantes, contribuindo para a construção de um panorama detalhado sobre a evolução dos indicadores nutricionais no Brasil e subsidiando futuras ações governamentais na área de saúde.

\section{Motivações e Justificativa}
\selectlanguage{brazil}
\textit{}
A análise do estado nutricional da população é um fator fundamental para a formulação de políticas públicas de saúde e assistência social. O Sistema de Vigilância Alimentar e Nutricional (\textit{Sisvan}) fornece dados essenciais sobre a condição antropométrica e hábitos alimentares da população atendida na Atenção Primária à Saúde (APS), permitindo identificar padrões de desnutrição, sobrepeso e obesidade.

Este projeto se justifica pela necessidade de aprimorar a análise desses dados para fornecer \textit{insights} que auxiliem profissionais de saúde e gestores públicos na tomada de decisão. Através da análise exploratória, modelagem de dados e visualização interativa, será possível entender padrões epidemiológicos e propor soluções para melhorar o acompanhamento nutricional da população brasileira.







\section{Objetivos}

Nosso objetivo com esse projeto é utilizar a análise dos dados do \textit{Sisvan} para identificar padrões de desnutrição, sobrepeso e obesidade, que são fatores críticos para a saúde pública. Outros objetivos específicos incluem: 
\begin{itemize}
\item Explorar e compreender a estrutura dos dados do \textit{Sisvan};

\item Analisar distribuições e tendências de indicadores nutricionais por região, idade e gênero;

\item Criar visualizações interativas para facilitar a análise dos dados;

\item Desenvolver modelos preditivos para identificar fatores de risco associados ao estado nutricional;

\item Gerar \textit{insights} que possam subsidiar políticas públicas de saúde alimentar e nutricional
\end{itemize}

\section{Descrição da Base de dados}

A base de dados utilizada neste projeto é proveniente do \textit{Sisvan}, mantido pelo Ministério da Saúde do Brasil. Os dados são anonimizados e contêm informações detalhadas sobre o estado nutricional de indivíduos atendidos na APS desde 2008. Os principais atributos disponíveis na base incluem:
\begin{itemize}
\item Identificação geográfica: Estado (\textit{SGUF}) e município (\textit{NOMUNICIPIO});

\item Dados demográficos: Idade (\textit{NUIDADEANO}), fase da vida (\textit{DSFASEVIDA}), sexo (\textit{SGSEXO}), raça/cor (\textit{DSRACACOR}) e comunidade/povo tradicional (\textit{DSPOVOCOMUNIDADE});

\item Nível educacional: Escolaridade do indivíduo (\textit{DSESCOLARIDADE});

\item Data e periodicidade do acompanhamento: Data do acompanhamento (\textit{DTACOMPANHAMENTO}) e competência do registro (\textit{NUCOMPETENCIA});

\item Medidas antropométricas: Peso (\textit{NUPESO}), altura (\textit{NUALTURA}), índice de massa corporal (\textit{DSIMC});

\item Classificações nutricionais: Estado nutricional para diferentes faixas etárias e critérios de análise (\textit{PESO X IDADE, PESO X ALTURA, CRIANÇA. ALTURA X IDADE, CRI. IMC X IDADE, ADULTO. ALTURA X IDADE, ADULTO. IMC X IDADE, COESTADONUTRIADULTO, COESTADONUTRIIDOSO});

\item Fonte do acompanhamento: Sistema de origem do dado (\textit{SISTEMAORIGEMACOMP}).
\end{itemize}
A base de dados possui registros individuais anonimizados, permitindo a análise de frequências relativas e prevalências dos estados nutricionais na população atendida. Os registros podem se sobrepor em casos de múltiplos acompanhamentos para um mesmo indivíduo, sendo recomendada a priorização dos dados provenientes do \textit{Sisvan} e Auxílio Brasil/Bolsa Família.

A análise desta base permitirá uma visão abrangente sobre a evolução do estado nutricional da população brasileira ao longo dos anos, auxiliando na formulação de estratégias de saúde pública.


\section{Referências bibliográficas}
Informações do \textit{Dataset}:
\href{https://dados.gov.br/dados/conjuntos-dados/sistema-de-vigilancia-alimentar-e-nutricional---sisvan}{dados.gov.br}

\end{document}



